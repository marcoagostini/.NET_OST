%Goals
%Sie kennen das Konzept von Klassen und Structs in C#
%Sie kennen den Parameterübergabe-Mechanismus von C#
%Sie knnen Properties und indexers
%Sie kennen das Konzept des Operator-Overloadings
%Sie könnne die gelernten Konzepte in praktischen Progammieraufgaben anwenden

\section{Klassen und Structs}

\subsection{Klassen}
Sind Referenz Typen und werden auf dem Heap angelegt.

\textbf{Vererbung:} Ableiten von Basisklasse möglich, Verwenden als Basisklasse möglich, Implementieren von Interfaces möglich.

\begin{lstlisting}
class Stack {
	int[] values;
	int top = 0;
	public Stack(int size) { /* ... */ }
	public void Push(int x) { /* ... */ }
	public int Pop() { /* ... */ } }
}
Stack s = new Stack(10); 
\end{lstlisting}

\subsection{Structs}
Sind Value types und werden auf dem Stack oder "in-line" in einem Objekt auf dem Heap abgelegt.

\textbf{Vererbung:} Ableiten von Basisklasse nicht möglich, Verwenden als Basisklasse auch nicht möglich, Implementieren von Interfaces ist möglich.

\begin{lstlisting}
struct Point {
int x;
int y;
public Point(int x, int y)
{
	this.x = x; this.y = y; 
} 
	public void MoveX(int x) { /* ... */ } 
	public void MoveY(int y) { /* ... */ }
}
Point p = new Point(2, 3); 
\end{lstlisting}

\textbf{Verwendung} \\
Ein Struct sollte nur unter folgenden Umständen verwendet werden.
\begin{itemize}
  \itemsep -0.5em 
  \item Repräsentiert einen einzelnen kleinen Wert
  \item 
\end{itemize}
